
\documentclass[12pt]{article}
%%%%%%%%%%%%%%%%%%%%%%%%%%%%%%%%%%%%%%%%%%%%%%%%%%%%%%%%%%%%%%%%%%%%%%%%%%%%%%%%%%%%%%%%%%%%%%%%%%%%%%%%%%%%%%%%%%%%%%%%%%%%%%%%%%%%%%%%%%%%%%%%%%%%%%%%%%%%%%%%%%%%%%%%%%%%%%%%%%%%%%%%%%%%%%%%%%%%%%%%%%%%%%%%%%%%%%%%%%%%%%%%%%%%%%%%%%%%%%%%%%%%%%%%%%%%
\usepackage{amsfonts}
\usepackage{amssymb}
\usepackage{sw20elba}

%TCIDATA{OutputFilter=LATEX.DLL}
%TCIDATA{Version=5.50.0.2890}
%TCIDATA{<META NAME="SaveForMode" CONTENT="1">}
%TCIDATA{BibliographyScheme=Manual}
%TCIDATA{Created=Monday, July 22, 2013 07:22:03}
%TCIDATA{LastRevised=Thursday, August 22, 2013 18:34:59}
%TCIDATA{<META NAME="GraphicsSave" CONTENT="32">}
%TCIDATA{<META NAME="DocumentShell" CONTENT="Articles\SW\mrvl">}
%TCIDATA{CSTFile=LaTeX article (bright).cst}
%TCIDATA{ComputeDefs=
%$\beta =0$
%$P=\left( 
%\begin{array}{ll}
%\alpha  & \beta  \\ 
%\omega \beta  & \alpha 
%\end{array}%
%\right) $
%}


\newtheorem{theorem}{Theorem}
\newtheorem{axiom}[theorem]{Axiom}
\newtheorem{claim}[theorem]{Claim}
\newtheorem{conjecture}[theorem]{Conjecture}
\newtheorem{corollary}[theorem]{Corollary}
\newtheorem{definition}[theorem]{Definition}
\newtheorem{example}[theorem]{Example}
\newtheorem{exercise}[theorem]{Exercise}
\newtheorem{lemma}[theorem]{Lemma}
\newtheorem{notation}[theorem]{Notation}
\newtheorem{problem}[theorem]{Problem}
\newtheorem{proposition}[theorem]{Proposition}
\newtheorem{remark}[theorem]{Remark}
\newtheorem{solution}[theorem]{Solution}
\newtheorem{summary}[theorem]{Summary}
\newenvironment{proof}[1][Proof]{\noindent\textbf{#1.} }{{\hfill $\Box$ \\}}
\input{tcilatex}
\addtolength{\textheight}{30pt}

\begin{document}

\title{Algebra 6.178}
\author{Michael Vaughan-Lee}
\date{July 2013}
\maketitle

Algebra 6.178 has presentation 
\[
\langle a,b,c\,|\,ca-bab,\,cb-\omega baa,\,pa-\lambda baa-\mu bab,\,pb-\nu
baa-\xi bab,\,pc,\,\text{class }3\rangle , 
\]%
where we write $A=\left( 
\begin{array}{ll}
\lambda & \mu \\ 
\nu & \xi%
\end{array}%
\right) $, and $A$ ranges over a set of representatives for the orbits of
non-singular $2\times 2$ matrices under the action 
\[
A\rightarrow \frac{1}{\det P}PAP^{-1} 
\]%
as $P$ ranges over non-singular matrices 
\[
P=\left( 
\begin{array}{ll}
\alpha & \beta \\ 
\pm \omega \beta & \pm \alpha%
\end{array}%
\right) . 
\]%
$\allowbreak $

These algebras are terminal unless $\xi =-\lambda $. The number of orbits of
non-singular matrices with $\xi =-\lambda $ is $(3p-1)/2$. The matrices
split up into one orbit of size $p-1$ (matrices $\left( 
\begin{array}{ll}
0 & y \\ 
\omega y & 0%
\end{array}%
\right) $), $p-1$ orbits of size $(p^{2}-1)/2$ (including two orbits of
elements $\left( 
\begin{array}{ll}
x & y \\ 
-\omega y & -x%
\end{array}%
\right) $), and $(p-1)/2$ orbits of size $p^{2}-1$. In all, 6.178 has $%
(3p^{2}-1)/2$ descendants of order $p^{7}$ and $p$-class 4. All orbits
contain matrices where $\lambda =0$ or $\lambda =1$.

It is possible to choose orbit representatives of the following 6 types: 

\begin{enumerate}
\item $\left( 
\begin{array}{ll}
0 & 1 \\ 
\omega  & 0%
\end{array}%
\right) $, 

\item $\left( 
\begin{array}{ll}
1 & 0 \\ 
0 & -1%
\end{array}%
\right) $ when $p=1\func{mod}4$, 

\item $\left( 
\begin{array}{ll}
0 & 1 \\ 
-\omega  & 0%
\end{array}%
\right) $ (all $\left( 
\begin{array}{ll}
0 & \mu  \\ 
-\omega \mu  & 0%
\end{array}%
\right) $ are in the same orbit as $\left( 
\begin{array}{ll}
0 & 1 \\ 
-\omega  & 0%
\end{array}%
\right) $, but this orbit also contains elements $\left( 
\begin{array}{ll}
1 & \mu  \\ 
-\omega \mu  & -1%
\end{array}%
\right) $), 

\item one representative $\left( 
\begin{array}{ll}
1 & \mu  \\ 
-\omega \mu  & -1%
\end{array}%
\right) $ $(\mu \neq 0)$ which is not in the same orbit as $\left( 
\begin{array}{ll}
0 & 1 \\ 
-\omega  & 0%
\end{array}%
\right) $ when $p=3\func{mod}4$, 

\item $p-3$ representatives $\left( 
\begin{array}{ll}
0 & \mu  \\ 
\nu  & 0%
\end{array}%
\right) $ $(\nu \neq \pm \omega \mu )$, and 

\item $(p-1)/2\,$\ representatives of the form $\left( 
\begin{array}{ll}
1 & \mu  \\ 
\nu  & -1%
\end{array}%
\right) $ $(\nu \neq -\omega \mu )$.
\end{enumerate}

We then obtain the following presentations for the descendants of 6.178. The
first four cases of the matrix $A$ are straightforward.

\[
\langle a,b,c\,|\,ca-bab,\,cb-\omega baa,\,pa-bab,\,pb-\omega
baa,\,pc-xbaaa,\,\text{class }4\rangle \,(\text{all }x,\,x\sim -x),
\]

\[
\langle a,b,c\,|\,ca-bab,\,cb-\omega baa,\,pa-baa,\,pb+bab,\,pc-xbaab,\,%
\text{class }4\rangle \,(\text{all }x,\,x\sim -x,\,p=1\func{mod}4),
\]

\[
\langle a,b,c\,|\,ca-bab,\,cb-\omega baa,\,pa-bab,\,pb+\omega
baa,\,pc-xbaaa,\,\text{class }4\rangle \,(\text{all }x,\,x\sim -x),
\]

For the one matrix $\left( 
\begin{array}{ll}
1 & \mu  \\ 
-\omega \mu  & -1%
\end{array}%
\right) $ $(\mu \neq 0)$ when $p=3\func{mod}4$, we have 
\[
\langle a,b,c\,|\,ca-bab,\,cb-\omega baa,\,pa-baa-\mu bab,\,pb+\omega \mu
baa+bab,\,pc-xbaaa,\,\text{class }4\rangle \,(\text{all }x,\,x\sim -x).
\]

For the $p-3$ matrices $A=\left( 
\begin{array}{ll}
0 & \mu \\ 
\nu & 0%
\end{array}%
\right) $ $(\nu \neq \pm \omega \mu )$, we have

\[
\langle a,b,c\,|\,ca-bab,\,cb-\omega baa,\,pa-\mu bab,\,pb-\nu
baa,\,pc-xbaaa,\,\text{class }4\rangle \,(\text{all }x,\,x\sim -x),
\]%
but we have extra descendants if $(\omega \mu +2\nu )(2\omega \nu +\mu
^{-1}\nu ^{2})$ is a square. If $\omega \mu +2\nu =0$ then we have,

\[
\langle a,b,c\,|\,ca-bab,\,cb-\omega baa,\,pa-\mu bab-xbaaa,\,pb-\nu
baa,\,pc,\,\text{class }4\rangle \,(x\neq 0,\,x\sim -x),
\]%
If $2\omega \nu +\mu ^{-1}\nu ^{2}=0$ we have

\[
\langle a,b,c\,|\,ca-bab,\,cb-\omega baa,\,pa-\mu bab,\,pb-\nu
baa-xbaaa,\,pc,\,\text{class }4\rangle \,(x\neq 0,\,x\sim -x),
\]%
and if $(\omega \mu +2\nu )(2\omega \nu +\mu ^{-1}\nu ^{2})=y^{2}\neq 0$
then for one such value $y$ we have

\[
\langle a,b,c\,|\,ca-bab,\,cb-\omega baa,\,pa-\mu bab,\,pb-\nu
baa-xbaaa,\,pc-ybaaa,\,\text{class }4\rangle \,(x\neq 0,\,x\sim -x).
\]

The situation is even more complicated for the $(p-1)/2$ matrices $A=\left( 
\begin{array}{ll}
1 & \mu \\ 
\nu & -1%
\end{array}%
\right) $ $(\nu \neq -\omega \mu )$. First we have

\[
\langle a,b,c\,|\,ca-bab,\,cb-\omega baa,\,pa-baa-\mu bab,\,pb-\nu
baa+bab,\,pc-xbaab,\,\text{class }4\rangle \,(\text{all }x). 
\]%
But if $\left( 1+\mu \nu \right) \left( 2\left( \omega \mu +\nu \right)
^{2}+\omega (1+\mu \nu )\right) $ is a square we have an additional $p-1$
descendants. It is not that easy to prove, but $\left( 1+\mu \nu \right)
\left( 2\left( \omega \mu +\nu \right) ^{2}+\omega (1+\mu \nu )\right) $
cannot equal zero, under the assumption that $A$ is not in the same orbit as
a matrix with $(1,1)$ entry equal to zero. If $\left( 1+\mu \nu \right)
\left( 2\left( \omega \mu +\nu \right) ^{2}+\omega (1+\mu \nu )\right)
=x^{2}\neq 0$, then if $x-\omega \mu -\nu =\omega \mu ^{2}+2\mu \nu +1=0$ we
have

\[
\langle a,b,c\,|\,ca-bab,\,cb-\omega baa,\,pa-baa-\mu bab-ybaab,\,pb-\nu
baa+bab,\,pc-xbaab,\,\text{class }4\rangle \,(y\neq 0,\,y\sim -y),
\]%
but if one of $x-\omega \mu -\nu $, $\omega \mu ^{2}+2\mu \nu +1$ is
non-zero we have

\[
\langle a,b,c\,|\,ca-bab,\,cb-\omega baa,\,pa-baa-\mu bab,\,pb-\nu
baa+bab-ybaab,\,pc-xbaab,\,\text{class }4\rangle \,(y\neq 0,\,y\sim -y).
\]%
And similarly for $-x$, if $x+\omega \mu +\nu =\omega \mu ^{2}+2\mu \nu +1=0$
we have

\[
\langle a,b,c\,|\,ca-bab,\,cb-\omega baa,\,pa-baa-\mu bab-ybaab,\,pb-\nu
baa+bab,\,pc+xbaab,\,\text{class }4\rangle \,(y\neq 0,\,y\sim -y),
\]%
but if one of $x+\omega \mu +\nu $, $\omega \mu ^{2}+2\mu \nu +1$ is
non-zero we have

\[
\langle a,b,c\,|\,ca-bab,\,cb-\omega baa,\,pa-baa-\mu bab,\,pb-\nu
baa+bab-ybaab,\,pc+xbaab,\,\text{class }4\rangle \,(y\neq 0,\,y\sim -y).
\]

There is a \textsc{Magma} program notes6.178.m which computes a
representative set of matrices $A$ for any given $p$, and then computes
representative values for the other parameters for each $A$.

\end{document}
