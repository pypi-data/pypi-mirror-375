\documentclass{article}
% common styling and macros shared by all proof files

\usepackage[top=1in, right=1in, left=1in, bottom=1.5in]{geometry}

\usepackage{amsmath,amsthm,amsfonts,amssymb,amscd}
\usepackage{listings}
\usepackage{hyperref}
\usepackage{xcolor}
\usepackage{xr}

\usepackage{enumerate} 
\usepackage{physics}
\usepackage{fancyhdr}
\usepackage{hyperref}
\usepackage{graphicx}
\usepackage{tcolorbox}
\usepackage{catchfile}
\usepackage{pdftexcmds}
\usepackage[T1]{fontenc}

% hyperref
\hypersetup{
  colorlinks=true,
  linkcolor=blue,
  linkbordercolor={0 0 1}
}

% \contrib macro to indicate inclusion in "contrib".
\usepackage{tcolorbox}
\newtcolorbox{warn_box}{colback=red!5!white,colframe=red!75!black}
\newcommand{\contrib}{{\begin{warn_box}This proof resides in \textbf{``contrib''} because it has not completed the vetting process.\end{warn_box}}} 
\newcommand{\floatingPoint}{{\begin{warn_box}This implementation is susceptible to floating-point vulnerabilities.\end{warn_box}}} 

% asOfCommit macro to version a code dependency. Arguments:
%    #1: relative path to file you are dependent on
%    #2: commit hash it was last edited. If outdated, this should be the second hash in the footnoote. Otherwise,
%            git log -n 1 --pretty=format:%h -- path/to/file.rs
\makeatletter
\ifnum\pdf@shellescape=1
   % "private" command that builds a link to a blob
  \newcommand{\linkOpendpBlob}[3]{%
    \href{https://github.com/opendp/opendp/blob/#1/#2#3}{\path{#3} at commit #1}}

  % latex macro expansion has a separate phase for \input evaluation
  %     immediately evaluate a command to write a temp file to ./out containing the current directory
  \immediate\write18{[ ! -f out/cwd.txt ] && (mkdir -p out && git rev-parse --show-prefix | sed "s|_|\@backslashchar\@backslashchar\@backslashchar_|g" > out/cwd.txt)}
  %     ...and then retrieve the current working directory by loading the temp file
  \CatchFileDef\GitWorkingDir{out/cwd.txt}{\endlinechar=-1}

  % command for building the (up to date) or (outdated) status
  \newcommand{\fileStatus}[2]{%
  \setbox0=\hbox{\input|"git --no-pager log -n1 --pretty='\@percentchar H' #1 | grep -E '^#2.*'"\unskip}\ifdim\wd0=0pt
        (outdated\footnote{See new changes with \texttt{git diff #2..\input|"git --no-pager log -n1 --pretty='\@percentchar h' #1" \GitWorkingDir\path{#1}}})\else
        (up to date)\fi
  }

  \newcommand{\asOfCommit}[2]{%
      % permalink the target
      \linkOpendpBlob{#2}{\GitWorkingDir}{#1}
      % conditionally add (outdated) or (up to date) depending on matching commit hash
      \fileStatus{#1}{#2}%
  }
\else
  % simplified command if shell-escape not enabled
  \newcommand{\asOfCommit}[2]{#1 at commit #2 (unknown status\footnote{Shell-escape is not enabled. Enable \texttt{--shell-escape} to check if this proof is up-to-date with the code.})}
\fi
\makeatother

% \vettingPR macro to link a PR. Arguments:
%    #1: PR number
\newcommand{\vettingPR}[1]{\href{https://github.com/opendp/opendp/pull/#1}{Pull Request \##1}}

% for links to rustdoc items in OpenDP. Arguments:
%    #1: path to item, and designation as trait, struct, fn, etc.
%    #2: item name
\makeatletter
\ifnum\pdf@shellescape=1
  % latex macro expansion has a separate phase for \input evaluation
  %     immediately evaluate a command to write a temp file to ./out containing the base path
  \immediate\write18{[ ! -f out/rustdoc.txt ] && mkdir -p out && ([ -z `kpsewhich --var-value OPENDP_RUSTDOC_PORT` ] && echo "https://docs.rs/opendp/`head -n 1 \@backslashchar`git rev-parse --show-toplevel\@backslashchar`/VERSION | sed 's|.*-dev.*|latest|g'`" || echo "http://localhost:`kpsewhich --var-value OPENDP_RUSTDOC_PORT`") > out/rustdoc.txt}
  %     ...and then retrieve the base path by loading the temp file
  \CatchFileDef\OpenDPRustdocBase{out/rustdoc.txt}{\endlinechar=-1}
\else
  % if shell commands are not enabled, just claim latest
  \newcommand{\OpenDPRustdocBase}{https://docs.rs/opendp/latest}
\fi
\makeatother
\newcommand{\rustdoc}[2]{\href{\OpenDPRustdocBase/opendp/#1.#2.html}{\texttt{#2}}}

% for links to external dependencies. Arguments:
%    #1: crate name
%    #2: path to item, and designation as trait, struct, fn, etc.
%    #3: item name
\newcommand{\docsrs}[3]{\href{https://docs.rs/#1/latest/#1/#2.#3.html}{\texttt{#3}}}

% minted (pseudocode)
\definecolor{codegreen}{rgb}{0,0.6,0}
\definecolor{codegray}{rgb}{0.5,0.5,0.5}
\definecolor{codepurple}{rgb}{0.58,0,0.82}
\definecolor{backcolour}{rgb}{0.95,0.95,0.92}

\lstdefinestyle{mystyle}{
    backgroundcolor=\color{backcolour},   
    commentstyle=\color{codegreen},
    keywordstyle=\color{magenta},
    numberstyle=\tiny\color{codegray},
    stringstyle=\color{codepurple},
    basicstyle=\ttfamily\footnotesize,
    breakatwhitespace=false,         
    breaklines=true,                 
    captionpos=b,                    
    keepspaces=true,                 
    numbers=left,                    
    numbersep=5pt,                  
    showspaces=false,                
    showstringspaces=false,
    showtabs=false,                  
    tabsize=2
}

\lstset{style=mystyle}

% common commands
\theoremstyle{definition}
\newtheorem{theorem}{Theorem}[section]
\newtheorem{lemma}[theorem]{Lemma}
\newtheorem{definition}[theorem]{Definition}
\newtheorem{warning}{Warning}
\newtheorem{corollary}{Corollary}
\newtheorem{proposition}{Proposition}
\newtheorem{remark}{Remark}
\newtheorem{observation}{Observation}
\newtheorem{note}{Note}

\newcommand{\vicki}[1]{{ {\color{olive}{(vicki)~#1}}}}
\newcommand{\hanwen}[1]{{ {\color{purple}{(hanwen)~#1}}}}
\newcommand{\zach}[1]{{ {\color{red}{(zach)~#1}}}}

\newcommand{\MultiSet}{\mathrm{MultiSet}}
\newcommand{\len}{\mathrm{len}}
\newcommand{\din}{\texttt{d\_in}}
\newcommand{\dout}{\texttt{d\_out}}
\newcommand{\T}{\texttt{T} }
\newcommand{\F}{\texttt{F} }
\newcommand{\Map}{\texttt{Map}}
\newcommand{\X}{\mathcal{X}}
\newcommand{\Y}{\mathcal{Y}}
\newcommand{\True}{\texttt{True}}
\newcommand{\False}{\texttt{False}}
\newcommand{\clamp}{\texttt{clamp}}
\newcommand{\function}{\texttt{function}}
\newcommand{\float}{\texttt{float }}
\newcommand{\questionc}[1]{\textcolor{red}{\textbf{Question:} #1}}


\newcommand{\validTransformation}[2]{%
  \begin{theorem}
  For every setting of the input parameters #1 to #2 such that the given preconditions
  hold, #2 raises an error (at compile time or run time) or returns a valid transformation. A valid transformation has the following properties:
  \begin{enumerate}
      \item \textup{(Data-independent runtime errors).}
      For every pair of members $x$ and $x'$ in \texttt{input\_domain},
      $\texttt{invoke}(x)$ and $\texttt{invoke}(x')$ either both return the same error or neither return an error. 

      \item \textup{(Appropriate output domain).} 
      For every member $x$ in \texttt{input\_domain}, $\function(x)$ is in \texttt{output\_domain} or raises a data-independent runtime error.
      
      \item \textup{(Stability guarantee).} 
      For every pair of members $x$ and $x'$ in \texttt{input\_domain} and for every pair $(\din, \dout)$, 
      where \din\ has the associated type for \texttt{input\_metric} and \dout\ has the associated type for \\ \texttt{output\_metric}, 
      if $x, x'$ are \din-close under \texttt{input\_metric}, $\texttt{stability\_map}(\din)$ does not raise an error,
      and $\texttt{stability\_map}(\din) = \dout$, 
      then $\function(x), \function(x')$ are $\dout$-close under \texttt{output\_metric}.
  \end{enumerate}
  \end{theorem}
}


\newcommand{\validMeasurement}[2]{%
  For every setting of the input parameters #1 to #2 such that the given preconditions
  hold, #2 raises an error (at compile time or run time) or returns a valid measurement. A valid measurement has the following properties:
  \begin{enumerate}
      \item \textup{(Data-independent runtime errors).}
      For every pair of members $x$ and $x'$ in \texttt{input\_domain},
      $\texttt{invoke}(x)$ and $\texttt{invoke}(x')$ either both return the same error or neither return an error. 

      \item \textup{(Privacy guarantee).}
      For every pair of members $x$ and $x'$ in \texttt{input\_domain} and for every pair $(\din, \dout)$,
      where \din\ has the associated type for \texttt{input\_metric} and \dout\ has the associated type for \\ \texttt{output\_measure},
      if $x, x'$ are \din-close under \texttt{input\_metric}, $\texttt{privacy\_map}(\din)$ does not raise an error,
      and $\texttt{privacy\_map}(\din) = \dout$,
      then $\function(x), \function(x')$ are $\dout$-close under \texttt{output\_measure}.
  \end{enumerate}
}

\newcommand{\validPostprocessor}[2]{%
  For every setting of the input parameters #1 to #2 such that the given preconditions
  hold, #2 raises an error (at compile time or run time) or returns a valid postprocessor. A valid postprocessor has the following property:
  \begin{enumerate}
      \item \textup{(Data-independent errors).}
      For every pair of members $x$ and $x'$ in \texttt{input\_domain},
      $\function(x), \function(x')$ either both raise the same error, 
      or neither raise an error.
  \end{enumerate}
}

\newcommand{\validOdometer}[2]{%
  For every setting of the input parameters #1 to #2 such that the given preconditions
  hold, #2 raises an error (at compile time or run time) or returns a valid odometer. 
  A valid odometer has the following properties:
  \begin{enumerate}
      \item \textup{(Data-independent runtime errors).}
      For every pair of members $x$ and $x'$ in \texttt{input\_domain},
      $\texttt{invoke}(x)$ and $\texttt{invoke}(x')$ either both return the same error or neither return an error. 

      \item \textup{(Valid odometer queryable).}
      For every member $x$ in \texttt{input\_domain},
      where $\function(x)$ does not raise an error,
      $\function(x)$ returns a valid odometer queryable.
  \end{enumerate}
}

\newcommand{\validOdometerQueryable}[2]{%
  For every setting of the input parameters #1 to #2 such that the given preconditions
  hold, #2 raises a data-independent error or returns a valid odometer queryable (\texttt{queryable}). 
  A valid odometer queryable is an \rustdoc{core/type}{OdometerQueryable} with the following properties:
  \begin{enumerate}
    \item \textup{(Data-independent errors).}
    For every pair of members $x$ and $x'$ in \texttt{input\_domain}, and every query $q$,
    $\texttt{queryable.invoke}_x(q)$ and $\texttt{queryable.invoke}_{x'}(q)$ either both return the same error or neither return an error. 

    \item \textup{(Privacy Guarantee).}
    For every pair of members $x$ and $x'$ in \texttt{input\_domain}, 
    every adversary $\mathcal{A}$, and for every pair $(\din, \dout)$, 
    where \din\ has the associated type for \texttt{input\_metric} and \dout\ has the associated type for \texttt{output\_measure}, 
    if $x$ and $x'$ are \din-close under \texttt{input\_metric}, $\texttt{queryable.eval(privacy\_loss)}$ does not return an error, 
    and $\texttt{queryable.eval(privacy\_loss)} = \dout$, 
    then $\texttt{View}(\mathcal{A} \leftrightarrow \texttt{queryable}_x), \texttt{View}(\mathcal{A} \leftrightarrow \texttt{queryable}_{x'})$ are $\dout$-close under \texttt{output\_measure},
    where $\texttt{queryable}_x$ denotes all messages received by $\mathcal{A}$ under $x$.
  \end{enumerate}
}


\title{\texttt{NoiseThresholdPrivacyMap<L01InfDistance<AbsoluteDistance<RBig>>, Approximate<MaxDivergence>> for ZExpFamily<1>}}
\author{Michael Shoemate}
\date{}

\begin{document}

\maketitle

\contrib
Proves soundness of the implementation of \rustdoc{measurements/noise_threshold/trait}{NoiseThresholdPrivacyMap} for \texttt{ZExpFamily<1>} in \asOfCommit{mod.rs}{f5bb719}.

\section{Hoare Triple}
\subsection*{Precondition}
\subsubsection*{Compiler-Verified}
\texttt{NoiseThresholdPrivacyMap} is parameterized as follows:
\begin{itemize}
    \item \texttt{MI}, the input metric, is of type \rustdoc{metrics/type}{L01InfDistance}\texttt{<AbsoluteDistance<RBig>>}
    \item \texttt{MO}, the output measure, is of type \rustdoc{measures/struct}{Approximate}\texttt{<MaxDivergence>}
\end{itemize}

\subsubsection*{User-Verified}
None

\subsection*{Pseudocode}
\lstinputlisting[language=Python,firstline=2,escapechar=|]{./pseudocode/NoiseThresholdPrivacyMap_for_ZExpFamily1.py}

\subsection*{Postcondition}
\begin{theorem}
    Given a distribution \texttt{self},
    returns \texttt{Err(e)} if \texttt{self} is not a valid distribution.
    Otherwise the output is \texttt{Ok(privacy\_map)}
    where \texttt{privacy\_map} observes the following:

    Define \texttt{function(x)} as a function that updates each pair $(k_i, v_i + Z_i)$,
    where $Z_i$ are iid samples from \texttt{self},
    and discards pairs where $v_i + Z_i$ has smaller magnitude than \texttt{|threshold|}.
    The ordering of returned pairs is independent from the input ordering.

    For every pair of elements $x, x'$ in \texttt{VectorDomain<AtomDomain<IBig>{}>},
    and for every pair (\texttt{d\_in}, \texttt{d\_out}),
    where \texttt{d\_in} has the associated type for \texttt{input\_metric} and \texttt{d\_out} has the associated type for \texttt{output\_measure},
    if $x, x'$ are \texttt{d\_in}-close under \texttt{input\_metric}, \texttt{privacy\_map(d\_in)} does not raise an exception,
    and $\texttt{privacy\_map(d\_in)} \leq \texttt{d\_out}$,
    then \texttt{function(x)}, \texttt{function(x')} are \texttt{d\_out}-close under \texttt{output\_measure}.
\end{theorem}

\begin{proof}
    Line \ref{line:noise-privacy-map} rejects \texttt{self} if \texttt{self} does not represent a valid distribution,
    satisfying the error conditions of the postcondition.

    We now construct the privacy map.
    Both the \texttt{l1} and \texttt{li} sensitivity can be floored, 
    as neighboring integer datasets always differ in whole integer increments.
    This doesn't affect the check on line \ref{line:li-check} to ensure \texttt{li} is non-negative.
    
    There are two ways to bound the l1 sensitivity given the three bounds:
    \begin{enumerate}
        \item The \texttt{l1} bound directly.
        \item Define $x \sim x'$ as $||x - x'||_0 \leq \texttt{l0}$ and $||x - x'||_\infty \leq \texttt{li}$. Then
        \begin{equation}
            \max_{x \sim x'} ||x - x'||_1 = \max{x \sim x'} \sum^n_i |x_i - x'_i| \leq \texttt{l0} \cdot \texttt{li}
        \end{equation}
    \end{enumerate}
    Line \ref{line:l1} updates \texttt{l1} to the tighter of these bounds.

    \texttt{li} similarly has multiple bounds:
    \begin{enumerate}
        \item The \texttt{li} bound directly.
        \item Define $x \sim x'$ as $||x - x'||_1 \leq \texttt{l1}$. Then
        \begin{equation}
            \max_{x \sim x'} ||x - x'||_\infty = \max{x \sim x'} \max_i |x_i - x'_i| \leq \max_i [0, \cdots, \texttt{l1}, \cdots, 0] = \texttt{l1}
        \end{equation}
    \end{enumerate}
    Line \ref{line:li} updates \texttt{li} to the tighter of these bounds.

    Now the \texttt{l1} sensitivity is zero if any of the three bounds are zero,
    due to the tightening of the \texttt{l1} bound on line \ref{line:l1}.
    Line \ref{line:zero-sens} then checks for zero sensitivity to return a privacy loss of zero epsilon, zero delta, 
    because all neighboring datasets have identical pairs, and the ordering is randomized.

    Otherwise sensitivity is nonzero, so if the scale is zero,
    the privacy loss is unbounded, shown on line \ref{line:zero-scale}.

    Now that the edge cases are handled, all sensitivities and scales are finite and strictly positive.

    The mechanism that adds noise to the values in the hashmap, as described in the conditions of the postcondition of \rustdoc{measurements/noise_threshold/trait}{NoiseThresholdPrivacyMap}, 
    matches the mechanism that adds noise to a vector, as described in the conditions of the postcondition of \rustdoc{measurements/noise/trait}{NoisePrivacyMap}.
    Therefore releasing the values incurs the privacy loss computed by \texttt{noise\_privacy\_map} as defined on line \ref{line:epsilon},
    The privacy map also ensures that \texttt{l1} is non-negative.

    We now focus on computing the remaining privacy loss parameter delta.
    To ensure that the distance to instability remains positive, 
    the threshold must be at least as large as the sensitivity, as checked on line \ref{line:threshold-check}.

    Adapting the proof from \cite{rogers2023unifyingprivacyanalysisframework} (Theorem 7).
    Consider $S$ to be the set of labels that are common between $x$ and $x'$.
    Define event $E$ to be any potential outcome of the mechanism for which all labels are in $S$
    (where only stable partitions are released).
    We then lower bound the probability of the mechanism returning an event $E$.
    In the following, $c_j$ denotes the exact count for partition $j$,
    and $Z_j$ is a random variable distributed according to \texttt{self}.

    \begin{align*}
        \Pr[E] &= \prod_{j \in x \backslash x'} \Pr[c_j + Z_j \le T] \\
        &\ge \prod_{j \in x \backslash x'} \Pr[\Delta_\infty + Z_j \le T] \\
        &\ge \Pr[\Delta_\infty + Z_j \le T]^{\Delta_0}
    \end{align*}

    The probability of returning a set of stable partitions ($\Pr[E]$) 
    is the probability of not returning any of the unstable partitions.
    We now solve for the choice of threshold $T$ such that $\Pr[E] \ge 1 - \delta$.

    \begin{align*}
        \Pr[\Delta_\infty + Z_j \le T]^{\Delta_0} &= \Pr[Z_j \le T - \Delta_\infty]^{\Delta_0} \\
        &= (1 - \Pr[Z_j > T - \Delta_\infty])^{\Delta_0}
    \end{align*}

    Let \texttt{d\_instability} denote the distance to instability of $T - \Delta_\infty$.

    Notice that when $C$ is a continuous laplace random variable and $D$ is a discrete laplace random variable,
    \begin{align}
        \Pr[C > t] &= \frac{e^{-t/s}}{2} > \frac{e^{-t/s}}{e^{1/s} + 1} = \Pr[D > t],
    \end{align}
    since $e^{1/s} > 1$.
    Unfortunately, the discrete laplace tail bound is not feasible to compute for large parameter choices,
    so we attempt to compute both tail bounds and take the minimum of the two.

    By the postconditions of \rustdoc{accuracy/tail_bounds/fn}{conservative\_discrete\_laplacian\_tail\_to\_alpha} \\
    and \rustdoc{accuracy/tail_bounds/fn}{conservative\_continuous\_laplacian\_tail\_to\_alpha},
    the tail mass is bounded above by both \\\texttt{alpha\_disc} and \texttt{alpha\_cont}.
    \texttt{delta\_single} is the smaller of the two, as both are conservative estimates.
    Both bounds are computed because while the discrete bound is more accurate, it can only be computed for relatively small scales.
    The continuous bound is a conservative estimate for larger scales that is always computable, 
    and converges to the discrete bound at higher scales.

    The probability that a random noise sample exceeds \texttt{d\_instability} is at most \texttt{delta\_single}.
    Therefore $\delta = 1 - (1 - \texttt{delta\_single})^{\Delta_0}$.

    The privacy map returns \texttt{(epsilon, delta.min(1))}, as $\delta$ is bounded by 1.
    It is shown that \function(x), \function(x') are \dout-close under \texttt{output\_measure}.
\end{proof}

\bibliographystyle{alpha}
\bibliography{ref}

\end{document}
